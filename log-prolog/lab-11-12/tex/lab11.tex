\textsc{\huge Лабораторная работа 11(1)} \\

Запустить среду Visual Prolog5.2. Настроить утилиту TestGoal (способ настройки см. в дополнительных материалах к лаб. раб.). Запустить тестовую программу, проанализировать реакцию системы и множество ответов. Разработать свою программу -- <<Телефонный справочник>>. Протестировать работу программы.

\begin{lstlisting}
domains
name, phoneNumber = symbol.

predicates
  usingNumber(name, phoneNumber).

clauses
  usingNumber("Tatiana", "88352318975").
  usingNumber("Sofia", "83452878650").
  usingNumber("Ekaterina", "84232958684").
  usingNumber("Irina", "842323318119").
  usingNumber("Egor", "83452878650").

goal
  usingNumber(Name, "88352318975");
  usingNumber(Name, "88005553535");
  usingNumber("Tatiana", "88352318975");
  usingNumber("Sofia", "88352318975");
  usingNumber("Sofia", PhoneNumber);
  usingNumber("Sofia", _).
\end{lstlisting}

~\newline

\textsc{\huge Лабораторная работа 11(2)} \\
Составить программу – базу знаний, с помощью которой можно определить, например, множество студентов, обучающихся в одном ВУЗе и их телефоны. Студент может одновременно обучаться в нескольких ВУЗах. Привести примеры возможных вариантов вопросов и варианты ответов (не менее 3-х). Описать порядок формирования вариантов
ответа. \\
Исходную базу знаний сформировать с помощью только фактов. \\
*Исходную базу знаний сформировать, используя правила. \\
**Разработать свою базу знаний (содержание произвольно).\\
\begin{lstlisting}
domains
  id = unsigned.
  name, uni = symbol.

predicates
  StudentId(id, name).
  StudyIn(id, uni).
  StudentsFromUniversity(uni, id, name).

clauses
  StudentId(0, "Tatiana").
  StudentId(1, "Ekaterina").
  StudentId(2, "Sofia").
  StudentId(3, "Irina").
  StudentId(4, "Egor").

  StudyIn(0, "BMSTU").
  StudyIn(1, "BMSTU").
  StudyIn(2, "BMSTU").
  StudyIn(3, "BMSTU").
  StudyIn(4, "BMSTU").
  StudyIn(0, "IKBFU").
  StudyIn(2, "Cambridge").
  StudyIn(3, "MIREA").

  StudentsFromUniversity(University, Id, Name) :- StudentId(Id, Name), StudyIn(Id, University).

goal
  % StudentsFromUniversity("BMSTU", Id, Name).
  StudentsFromUniversity(Uni, Id, "Tatiana").
\end{lstlisting}