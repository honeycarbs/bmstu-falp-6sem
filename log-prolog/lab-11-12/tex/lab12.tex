\textsc{\huge Лабораторная работа 12(1)} \\
Составить программу, т.е. модель предметной области – базу знаний, объединив в ней информацию -- знания:
\begin{itemize}
	\item <<Телефонный справочник>>: Фамилия, №тел, Адрес -- структура (Город, Улица, №дома, №кв),
	\item <<Автомобили>>: Фамилия\_владельца, Марка, Цвет, Стоимость, и др.,
	\item <<Вкладчики банков>>: Фамилия, Банк, счет, сумма, др.
\end{itemize}
Владелец может иметь несколько телефонов, автомобилей, вкладов (Факты). Используя правила, обеспечить возможность поиска:
\begin{enumerate}
	\item \begin{tasks}[label=\asbuk*)](1)
		\task По № телефона найти: Фамилию, Марку автомобиля, Стоимость автомобиля (может быть несколько);
		\task  Используя сформированное в пункте а) правило, по № телефона найти: только Марку автомобиля (автомобилей может быть несколько).
	\end{tasks}
	\item Используя простой, не составной вопрос: по Фамилии (уникальна в городе, но в разных городах есть однофамильцы) и Городу проживания найти:
	Улицу проживания, Банки, в которых есть вклады и №телефона.
\end{enumerate}

\begin{lstlisting}
domains
  name = symbol.
  phoneNumber = symbol.
  city = symbol.
  street = symbol.
  houseNumber = integer.
  flatNumber = integer.
  fullAddress = address(city, street, houseNumber, flatNumber)
  carBrand = symbol.
  carColor = symbol.
  carCost = integer.
  bankName = symbol.
  bankAccount = symbol.
  accountCost = integer.
  predicates
  UsingNumber(name, phoneNumber, fullAddress).
  UsingCar(name, carBrand, carColor, carCost).
  BankDepositor(name, bankName, bankAccount, accountCost).
  FindPersonByNumber(phoneNumber, name, carBrand, carCost).
  FindCarBrandByNumber(phoneNumber, carBrand).
  FindPersonByNameCity(name, city, phoneNumber, street, bankName).
  FindPersonByBrandColor(carBrand, carColor, name, city, phoneNumber, bankName).

clauses
  UsingNumber("Kazaeva", "88126152221", address("Moscow", "Baumanskaya st.", 69, 1337)).
  UsingNumber("Kazaeva", "88123616141", address("St. Petersburg", "Not Baumansksya st.", 68, 1336)).
  UsingNumber("Paraskun", "83452878650", address("Moscow", "Ladojskaya st.", 1488, 666)).
  UsingNumber("Suprunova", "88638423840", address("Moscow", "Novaya Doroga st.", 999, 666)).
  UsingNumber("Alferova", "84232958684", address("Moscow", "Bakuninskaya st.", 969, 696)).
  UsingCar("Kazaeva", "BMW", "Pink", 35700).
  UsingCar("Suprunova", "Cadillac", "Black", 36895).
  UsingCar("Kazaeva", "Ford", "White", 46190).
  UsingCar("Suprunova", "Ferrari", "Red", 625000).
  UsingCar("Alferova", "Ferrari", "Red", 625000).
  BankDepositor("Paraskun", "Not Sberbank", "07279163", 6900000).
  BankDepositor("Kazaeva", "Tinkoff", "50679823", 3).
  BankDepositor("Alferova", "VTB", "41572869", 300000).
  BankDepositor("Suprunova", "Dorama Bank", "92033800", 50000).
  FindPersonByNumber(PhoneNumber, Name, CarBrand, CarCost) :- UsingNumber(Name, PhoneNumber, _), UsingCar(Name, CarBrand, _, CarCost).
  FindCarBrandByNumber(PhoneNumber, CarBrand) :- FindPersonByNumber(PhoneNumber, _, CarBrand, _).
  FindPersonByNameCity(Name, City, PhoneNumber, Street, BankName) :- UsingNumber(Name, PhoneNumber, address(City, Street, _, _)), BankDepositor(Name, BankName, _, _).
  FindPersonByBrandColor(CarBrand, CarColor, Name, City, PhoneNumber, BankName) :- UsingCar(Name, CarBrand, CarColor, _), UsingNumber(Name, PhoneNumber, address(City, _, _, _)), BankDepositor(Name, BankName, _, _).
goal
  FindPersonByNumber("88123616141", Name, CarBrand, CarCost). % 1(A)
  FindCarBrandByNumber("88638423840", CarBrand). % 1 (B)
\end{lstlisting}

Для одного из вариантов ответов, и для а) и для б), описать словесно порядок поиска ответа на вопрос, указав, как выбираются знания, и, при этом, для каждого этапа унификации, выписать подстановку -- наибольший общий унификатор, и соответствующие примеры термов.

\begin{sidewaystable}
	\medskip
	\resizebox{\linewidth}{!}{%
		\tabcolsep=2pt
		\begin{tabular}{|ccc|}
			\hline
			\multicolumn{1}{|c|}{\# шага} & \multicolumn{1}{c|}{\begin{tabular}[c]{@{}c@{}}Сравниваемые термы; результат;\\ Подстановка, если есть\end{tabular}} & \begin{tabular}[c]{@{}c@{}}Дальнейшие действия: прямой ход или\\ откат (к чему приводит?)\end{tabular} \\ \hline
			\multicolumn{1}{|c|}{1} & \multicolumn{1}{c|}{\begin{tabular}[c]{@{}c@{}}Сравнение \\ FindPersonByNumber("88123616141", Name, CarBrand, CarCost). \\ и UsingNumber("Kazaeva", "88126152221", address("Moscow", "Baumanskaya st.", 69, 1337)). \\ Унификация неуспешна.\end{tabular}} & \begin{tabular}[c]{@{}c@{}}Прямой ход; переход к следующему\\ предложению.\end{tabular} \\ \hline
			\multicolumn{3}{|c|}{...} \\ \hline
			\multicolumn{1}{|c|}{14} & \multicolumn{1}{c|}{\begin{tabular}[c]{@{}c@{}}Сравнение FindPersonByNumber("88123616141", Name, CarBrand, CarCost). и \\ FindPersonByNumber(PhoneNumber, Name, CarBrand, CarCost)\\ Унификация успешна.\\ Подстановка:\{Phone\_number =88123616141,\\ Name = Name, CarBrand = CarBrand,\\ CarCost = CarCost\}\end{tabular}} & \begin{tabular}[c]{@{}c@{}}Прямой ход; переход к терму правила.\\ Унификация\\ UsingNumber(Name, "88123616141", \_)\end{tabular} \\ \hline
			\multicolumn{1}{|c|}{15} & \multicolumn{1}{c|}{\begin{tabular}[c]{@{}c@{}}Сравнение\\ UsingNumber(Name, "88123616141", \_)\\ UsingNumber("Kazaeva", "88126152221", address("Moscow", "Baumanskaya st.", 69, 1337)).\\ Унификация неуспешна.\end{tabular}} & \begin{tabular}[c]{@{}c@{}}Прямой ход; переход к следующему\\ предложению.\end{tabular} \\ \hline
			\multicolumn{1}{|c|}{16} & \multicolumn{1}{c|}{\begin{tabular}[c]{@{}c@{}}Сравнение\\ UsingNumber(Name, "88123616141", \_)\\ UsingNumber("Kazaeva", "88123616141", address("St. Petersburg", "Not Baumansksya st.", 68, 1336)).\\ Унификация успешна. \\ Подстановка: \{Name="Kazaeva"\}\end{tabular}} & \begin{tabular}[c]{@{}c@{}}Прямой ход; переход к следующем терму\\ правила\\ UsingCar(“Kazaeva”, CarBrand, \_, CarCost).\end{tabular} \\ \hline
			\multicolumn{1}{|c|}{17} & \multicolumn{1}{c|}{\begin{tabular}[c]{@{}c@{}}Сравнение UsingCar(“Kazaeva”, CarBrand, \_, CarCost). и\\ UsingNumber("Kazaeva", "88126152221", address("Moscow", "Baumanskaya st.", 69, 1337)).\\ Унификация неуспешна.\end{tabular}} & \begin{tabular}[c]{@{}c@{}}Прямой ход; переход к следующему\\ предложению.\end{tabular} \\ \hline
			\multicolumn{3}{|c|}{...} \\ \hline
			\multicolumn{1}{|c|}{22} & \multicolumn{1}{c|}{\begin{tabular}[c]{@{}c@{}}Сравнение UsingCar(“Kazaeva”, CarBrand, \_, CarCost). и\\ UsingCar("Kazaeva", "BMW", "Pink", 35700).\\ Унификация успешна.\\ Подстановка \{CarBrand = "BMW",\\ CarCost = 35700\}\end{tabular}} & \begin{tabular}[c]{@{}c@{}}Сохранение подстановки \{Name =\\ “Kazaeva”, CarBrand = "BMW",\\ CarCost = 35700\} в памяти.\\ Реконкретизация CarBrand и CarCost.\\ Прямой ход; переход к следующему\\ предложению.\end{tabular} \\ \hline
			\multicolumn{3}{|c|}{...} \\ \hline
			\multicolumn{1}{|c|}{28} & \multicolumn{1}{c|}{\begin{tabular}[c]{@{}c@{}}Сравнение UsingCar(“Kazaeva”, CarBrand, \_, CarCost). и UsingCar("Kazaeva", "Ford", "White", 46190).\\ Унификация успешна.\\ Подстановка \{CarBrand = "Ford",\\ CarCost = 46190\}\end{tabular}} & \begin{tabular}[c]{@{}c@{}}Сохранение подстановки \{Name =\\ “Kazaeva”,\{Car\_brand = "Ford",\\ Car\_cost = 46190\}\} в памяти.\\ Реконкретизация CarBrand и CarCost.\\ Прямой ход; переход к следующему\\ предложению.\end{tabular} \\ \hline
			\multicolumn{1}{|c|}{29} & \multicolumn{1}{c|}{\begin{tabular}[c]{@{}c@{}}Сравнение UsingCar(“Kazaeva”, CarBrand, \_, CarCost). и UsingCar("Suprunova", "Ferrari", "Red", 625000).\\ Унификация неуспешна.\end{tabular}} & \begin{tabular}[c]{@{}c@{}}Прямой ход; переход к следующему\\ предложению.\end{tabular} \\ \hline
		\end{tabular}
	}
\end{sidewaystable}

\begin{sidewaystable}
	\medskip
	\resizebox{\linewidth}{!}{%
		\tabcolsep=2pt
		\begin{tabular}{|ccc|}
			\hline
			\multicolumn{1}{|c|}{\# шага} & \multicolumn{1}{c|}{\begin{tabular}[c]{@{}c@{}}Сравниваемые термы; результат;\\ Подстановка, если есть\end{tabular}} & \begin{tabular}[c]{@{}c@{}}Дальнейшие действия: прямой ход или\\ откат (к чему приводит?)\end{tabular} \\ \hline
			\multicolumn{3}{|c|}{...} \\ \hline
			\multicolumn{1}{|c|}{37} & \multicolumn{1}{c|}{\begin{tabular}[c]{@{}c@{}}Сравнение UsingCar(“Kazaeva”, CarBrand, \_, CarCost). и \\ FindPersonByBrandColor(CarBrand, CarColor, Name, City, PhoneNumber, BankName).\\ Унификация неуспешна.\end{tabular}} & \begin{tabular}[c]{@{}c@{}}Обратный ход. Реконкретизация Name.\\ Переход к следующему предложению.\end{tabular} \\ \hline
			\multicolumn{3}{|c|}{...} \\ \hline
			\multicolumn{1}{|c|}{38} & \multicolumn{1}{c|}{\begin{tabular}[c]{@{}c@{}}Сравнение\\ UsingNumber(Name, "88123616141", \_)\\ UsingNumber("Paraskun", "83452878650", address("Moscow", "Ladojskaya st.", 1488, 666)).\\ Унификация неуспешна.\end{tabular}} & \begin{tabular}[c]{@{}c@{}}Обратный ход.; переход к следующему\\ предложению.\end{tabular} \\ \hline
			\multicolumn{3}{|c|}{...} \\ \hline
			\multicolumn{1}{|r|}{50} & \multicolumn{1}{l|}{\begin{tabular}[c]{@{}l@{}}Сравнение FindPersonByNumber("88123616141", Name, CarBrand, CarCost). и\\ FindPersonByBrandColor(CarBrand, CarColor, Name, City, PhoneNumber, BankName).\\ Унификация неуспешна.\end{tabular}} & \multicolumn{1}{l|}{\begin{tabular}[c]{@{}l@{}}Обратный ход.; переход к следующему\\ предложению.\end{tabular}} \\ \hline
			\multicolumn{1}{|c|}{51} & \multicolumn{1}{c|}{\begin{tabular}[c]{@{}c@{}}Сравнение FindPersonByNumber("88123616141", Name, CarBrand, CarCost). и\\ FindCarBrandByNumber(PhoneNumber, CarBrand)\\ Унификация неуспешна.\end{tabular}} & \begin{tabular}[c]{@{}c@{}}Прямой ход; переход к следующему\\ предложению.\end{tabular} \\ \hline
			\multicolumn{3}{|c|}{...} \\ \hline
			\multicolumn{1}{|c|}{54} & \multicolumn{1}{c|}{\begin{tabular}[c]{@{}c@{}}Сравнение FindPersonByNumber("88123616141", Name, CarBrand, CarCost). и\\ FindCarBrandByNumber(PhoneNumber, CarBrand) и\\ FindPersonByBrandColor(CarBrand, CarColor, Name, City, PhoneNumber, BankName)\end{tabular}} & \begin{tabular}[c]{@{}c@{}}Прямой ход; переход к вопросу.\\ Отображение результата.\end{tabular} \\ \hline
		\end{tabular}
	}
\end{sidewaystable}

\begin{sidewaystable}
	\medskip
	\resizebox{\linewidth}{!}{%
		\tabcolsep=2pt
		\begin{tabular}{|ccc|}
			\hline
			\multicolumn{1}{|c|}{} & \multicolumn{1}{c|}{\begin{tabular}[c]{@{}c@{}}Сравниваемые термы; результат;\\ Подстановка, если есть\end{tabular}} & \begin{tabular}[c]{@{}c@{}}Дальнейшие действия: прямой ход или\\ откат (к чему приводит?)\end{tabular} \\ \hline
			\multicolumn{1}{|c|}{1} & \multicolumn{1}{c|}{\begin{tabular}[c]{@{}c@{}}Сравнение FindCarBrandByNumber("88638423840", CarBrand). и\\ UsingNumber("Kazaeva", "88126152221", address("Moscow", "Baumanskaya st.", 69, 1337)).\\ Унификация неуспешна.\end{tabular}} & \begin{tabular}[c]{@{}c@{}}Прямой ход; переход к следующему\\ предложению.\end{tabular} \\ \hline
			\multicolumn{3}{|c|}{...} \\ \hline
			\multicolumn{1}{|c|}{15} & \multicolumn{1}{c|}{\begin{tabular}[c]{@{}c@{}}Сравнение FindCarBrandByNumber("88638423840", CarBrand). и\\ FindCarBrandByNumber(PhoneNumber, CarBrand)\\ Унификация успешна.\\ Подстановка:\{PhoneNumber =88638423840,\\ Name = Name, CarBrand = CarBrand\}\end{tabular}} & \begin{tabular}[c]{@{}c@{}}Прямой ход; переход к терму правила.\\ Унификация\\ FindPersonByNumber(88638423840, \_, CarBrand, \_).\end{tabular} \\ \hline
			\multicolumn{1}{|c|}{16} & \multicolumn{1}{c|}{\begin{tabular}[c]{@{}c@{}}Сравнение FindPersonByNumber(88638423840, \_, CarBrand, \_). \\ и UsingNumber("Kazaeva", "88126152221", address("Moscow", "Baumanskaya st.", 69, 1337)).\\ Унификация неуспешна.\end{tabular}} & \begin{tabular}[c]{@{}c@{}}Прямой ход; переход к следующему\\ предложению.\end{tabular} \\ \hline
			\multicolumn{3}{|c|}{...} \\ \hline
			\multicolumn{1}{|c|}{27} & \multicolumn{1}{c|}{\begin{tabular}[c]{@{}c@{}}Сравнение FindPersonByNumber(88638423840, \_, CarBrand, \_).  и \\ FindPersonByNumber(88638423840, Name, CarBrand, CarCost). \\ Унификация успешна.\\ Подстановка: \{Car\_brand = Car\_brand\}\end{tabular}} & \begin{tabular}[c]{@{}c@{}}Прямой ход; переход к терму правила.\\ Унификация\\ FindPersonByNumber(88638423840, \_,\\ CarBrand, \_)\end{tabular} \\ \hline
			\multicolumn{1}{|c|}{28} & \multicolumn{1}{c|}{\begin{tabular}[c]{@{}c@{}}Сравнение UsingNumber(Name, "88638423840", \_) и\\ UsingNumber("Kazaeva", "88126152221", address("Moscow", "Baumanskaya st.", 69, 1337)).\\ Унификация неуспешна.\end{tabular}} & \begin{tabular}[c]{@{}c@{}}Прямой ход; переход к следующему\\ предложению.\end{tabular} \\ \hline
			\multicolumn{3}{|c|}{...} \\ \hline
			\multicolumn{1}{|c|}{32} & \multicolumn{1}{c|}{\begin{tabular}[c]{@{}c@{}}Сравнение  UsingNumber(Name, "88638423840", \_)  и\\ UsingNumber("Suprunova", "88638423840", address("Moscow", "Novaya Doroga st.", 999, 666)).\\ Унификация успешна.\\ Подстановка: \{Surname=”Suprunova”\}\end{tabular}} & \begin{tabular}[c]{@{}c@{}}Прямой ход; переход к следующему\\ предложению.\end{tabular} \\ \hline
			\multicolumn{3}{|c|}{...} \\ \hline
			\multicolumn{1}{|c|}{44} & \multicolumn{1}{c|}{\begin{tabular}[c]{@{}c@{}}Сравнение UsingNumber("Suprunova", "88638423840", \_)  и\\ fFindPersonByBrandColor(CarBrand, CarColor, Name, City, PhoneNumber, BankName)\\ Унификация неуспешна.\end{tabular}} & \begin{tabular}[c]{@{}c@{}}Обратный ход; переход к следующему\\ терму правила\\ UsingCar(“Suprunova”, CarBrand, \_, CarCost)\end{tabular} \\ \hline
			\multicolumn{1}{|c|}{45} & \multicolumn{1}{c|}{\begin{tabular}[c]{@{}c@{}}Сравнение UsingCar(“Suprunova”, CarBrand, \_, CarCost) и\\ UsingNumber("Kazaeva", "88126152221", address("Moscow", "Baumanskaya st.", 69, 1337)).\\ Унификация неуспешна.\end{tabular}} & \begin{tabular}[c]{@{}c@{}}Прямой ход; переход к следующему\\ предложению.\end{tabular} \\ \hline
			\multicolumn{3}{|c|}{...} \\ \hline
			\multicolumn{1}{|c|}{62} & \multicolumn{1}{c|}{\begin{tabular}[c]{@{}c@{}}Сравнение UsingCar(“Suprunova”, CarBrand, \_, CarCost) и\\ FindPersonByBrandColor(CarBrand, CarColor, Name, City, PhoneNumber, BankName)\\ Унификация неуспешна.\end{tabular}} & \begin{tabular}[c]{@{}c@{}}Обратный ход. Реконкретизация Name.\\ Переход к следующему предложению.\end{tabular} \\ \hline
			\multicolumn{1}{|c|}{63} & \multicolumn{1}{c|}{\begin{tabular}[c]{@{}c@{}}Сравнение UsingCar(“Suprunova”, CarBrand, \_, CarCost) и\\ UsingCar("Suprunova", "Cadillac", "Black", 36895).\\ Унификация успешна. Подстановка: \{CarBrand =\\ "Cadillac", CarCost = 36895\}\end{tabular}} & \begin{tabular}[c]{@{}c@{}}Сохранение подстановки \{CarBrand =\\ "Cadillac", CarCost = 36895\} в памяти; прямой ход;\\ Реконкретизация Car\_brand, Car\_cost.\end{tabular} \\ \hline
			\multicolumn{3}{|c|}{...} \\ \hline
			\multicolumn{1}{|c|}{65} & \multicolumn{1}{c|}{\begin{tabular}[c]{@{}c@{}}Сравнение UsingCar(“Suprunova”, CarBrand, \_, CarCost) и\\ UsingCar("Suprunova", "Ferrari", "Red", 625000).Унификация успешна.\\ Подстановка: \{CarBrand = "Ferrari", CarColor = "Red",\\ Car\_cost = 625000\}\end{tabular}} & \begin{tabular}[c]{@{}c@{}}Сохранение подстановки: \{CarBrand = "Ferrari", CarColor = "Red",Car\_cost = 625000\}\\ в памяти; прямой ход;\\ Реконкретизация Car\_brand, Car\_cost.\end{tabular} \\ \hline
		\end{tabular}
	}
\end{sidewaystable}

\begin{sidewaystable}
	\medskip
	\resizebox{\linewidth}{!}{%
		\tabcolsep=2pt
		\begin{tabular}{|ccc|}
			\hline
			\multicolumn{1}{|c|}{} & \multicolumn{1}{c|}{\begin{tabular}[c]{@{}c@{}}Сравниваемые термы; результат;\\ Подстановка, если есть\end{tabular}} & \begin{tabular}[c]{@{}c@{}}Дальнейшие действия: прямой ход или\\ откат (к чему приводит?)\end{tabular} \\ \hline
			\multicolumn{1}{|c|}{66} & \multicolumn{1}{c|}{\begin{tabular}[c]{@{}c@{}}Сравнение UsingCar(“Suprunova”, CarBrand, \_, CarCost)  и\\ FindPersonByBrandColor(CarBrand, CarColor, Name, City, PhoneNumber, BankName) \\ Унификация неуспешна.\end{tabular}} & \begin{tabular}[c]{@{}c@{}}Обратный ход; возвращение к терму\\ FindPersonByNumber(PhoneNumber, \_, CarBrand, \_).\\ Реконкретизация Name в обеих\\ подстановках.\end{tabular} \\ \hline
			\multicolumn{3}{|c|}{...} \\ \hline
			\multicolumn{1}{|c|}{68} & \multicolumn{1}{c|}{\begin{tabular}[c]{@{}c@{}}Сравнение find\_car\_brand\_by\_number("+7(984)874-\\ 91-23", Car\_brand)  и\\ FindPersonByBrandColor(CarBrand, CarColor, Name, City, PhoneNumber, BankName)\\ Унификация неуспешна.\end{tabular}} & \begin{tabular}[c]{@{}c@{}}Обратный ход; возвращение к вопросу.\\ Восстановление результата из памяти.\end{tabular} \\ \hline
		\end{tabular}
	}
\end{sidewaystable}
\textsc{\huge Лабораторная работа 12(2)} \\
Составить программу, объединив в ней информацию-знания (12.1).
По Марке и Цвету автомобиля найти Фамилию, Город, Телефон и Банки, в которых владелец автомобиля имеет вклады. \\
Владельца может быть несколько (не более трех), один и ни одного. \\
\begin{tasks}[label=\arabic*]
	\task Для каждого из трех вариантов подробно описать порядок формирования ответа в виде таблицы. При этом указать -- отметить моменты очередного запуска алгоритма унификации и полный результат его работы. Обосновать следующий шаг работы системы. Выписать унификаторы – подстановки. Указать моменты, причины и результат отката, если он есть.
	\task Для случая нескольких владельцев (2-х). Приведите пример (таблицы)
	работы системы при разных порядках следования в БЗ процедур, и знаний в них (<<Телефонный справочник>>, <<Автомобили>>, <<Вкладчики банков>> или <<Автомобили>>, <<Вкладчики банков>>, <<Телефонный справочник>>). Сделать вывод: одинаковы ли множество работ и объем в разных случаях.
	\task Оформите 2 таблицы, демонстрирующие порядок работы алгоритма
	унификации вопроса и подходящего заголовка правила (для двух случаев из пункта 2) и укажите результаты его работы: ответ и побочный
	эффект.
\end{tasks}
\begin{lstlisting}
goal
  FindPersonByBrandColor("BMW", "Black", Name, City, PhoneNumber, BankName). % (no solutions)
  FindPersonByBrandColor("Ford", "White", Name, City, PhoneNumber, BankName). % 1 solution
  FindPersonByBrandColor("Ferrari", "Red", Name, City, PhoneNumber, BankName). % 2 solutions
\end{lstlisting}


\begin{sidewaystable}
	\medskip
	\resizebox{\linewidth}{!}{%
		\tabcolsep=2pt
		\begin{tabular}{|ccc|}
			\hline
			\multicolumn{1}{|l|}{\# шага} & \multicolumn{1}{l|}{\begin{tabular}[c]{@{}l@{}}Сравниваемые термы; результат;\\ Подстановка, если есть\end{tabular}} & \multicolumn{1}{l|}{\begin{tabular}[c]{@{}l@{}}Дальнейшие действия: прямой ход или\\ откат (к чему приводит?)\end{tabular}} \\ \hline
			\multicolumn{1}{|r|}{1} & \multicolumn{1}{l|}{\begin{tabular}[c]{@{}l@{}}Сравнение FindPersonByBrandColor("BMW", "Black", Name, City, PhoneNumber, BankName). и\\ phone("Pavlov", "+7(934)245-34-12",\\ address("Moscow", "St.1905 year", 20, 154))\\ Унификация неуспешна.\end{tabular}} & \multicolumn{1}{l|}{\begin{tabular}[c]{@{}l@{}}Прямой ход; переход к следующему\\ предложению.\end{tabular}} \\ \hline
			\multicolumn{3}{|l|}{...} \\ \hline
			\multicolumn{1}{|r|}{17} & \multicolumn{1}{l|}{\begin{tabular}[c]{@{}l@{}}Сравнение FindPersonByBrandColor("BMW", "Black", Name, City, PhoneNumber, BankName) \\ и FindPersonByBrandColor(CarBrand, CarColor, Name, City, PhoneNumber, BankName)\\ Унификация успешна.\\ Подстановка:\{ CarBrand = “BMW”,\\ CarColor = "Black"\}\end{tabular}} & \multicolumn{1}{l|}{Прямой ход; переход к терму правилa UsingCar(Name, CarBrand, CarColor, \_)} \\ \hline
			\multicolumn{1}{|r|}{18} & \multicolumn{1}{l|}{\begin{tabular}[c]{@{}l@{}}Сравнение UsingCar(Name, “BMW”, "Black", \_)\\ и \\ UsingNumber("Kazaeva", "88126152221", address("Moscow", "Baumanskaya st.", 69, 1337)).\\ Унификация неуспешна.\end{tabular}} & \multicolumn{1}{l|}{\begin{tabular}[c]{@{}l@{}}Прямой ход; переход к следующему\\ предложению.\end{tabular}} \\ \hline
			\multicolumn{3}{|l|}{...} \\ \hline
			\multicolumn{1}{|r|}{23} & \multicolumn{1}{l|}{\begin{tabular}[c]{@{}l@{}}Сравнение UsingCar(Name, “BMW”, "Black", \_)\\ и \\ UsingCar("Kazaeva", "BMW", "Pink", 35700).\\ Унификация неуспешна.\end{tabular}} & \multicolumn{1}{l|}{\begin{tabular}[c]{@{}l@{}}Прямой ход; переход к следующему\\ предложению.\end{tabular}} \\ \hline
			\multicolumn{3}{|l|}{...} \\ \hline
			\multicolumn{1}{|r|}{34} & \multicolumn{1}{l|}{\begin{tabular}[c]{@{}l@{}}Сравнение UsingCar(Name, “BMW”, "Black", \_)\\ и FindPersonByBrandColor(CarBrand, CarColor, Name, City, PhoneNumber, BankName).\\ Унификация неуспешна.\end{tabular}} & \multicolumn{1}{l|}{\begin{tabular}[c]{@{}l@{}}Обратный ход до правила; вновь откат\\ ввиду невыполнения первого условия\\ конъюнкции.\end{tabular}} \\ \hline
			\multicolumn{1}{|r|}{35} & \multicolumn{2}{l|}{Выдать сообщение “No solution”} \\ \hline
		\end{tabular}
	}
\end{sidewaystable}

\begin{sidewaystable}
	\medskip
	\resizebox{\linewidth}{!}{%
		\tabcolsep=2pt
		\begin{tabular}{|ccc|}
			\hline
			\multicolumn{1}{|l|}{\# шага} & \multicolumn{1}{l|}{\begin{tabular}[c]{@{}l@{}}Сравниваемые термы; результат;\\ Подстановка, если есть\end{tabular}} & \multicolumn{1}{l|}{\begin{tabular}[c]{@{}l@{}}Дальнейшие действия: прямой ход или\\ откат (к чему приводит?)\end{tabular}} \\ \hline
			\multicolumn{1}{|r|}{1} & \multicolumn{1}{l|}{\begin{tabular}[c]{@{}l@{}}Сравнение FindPersonByBrandColor("Ford", "White", Name, City, PhoneNumber, BankName). \\ и\\ UsingNumber("Kazaeva", "88126152221", address("Moscow", "Baumanskaya st.", 69, 1337)).\\ Унификация неуспешна.\end{tabular}} & \multicolumn{1}{l|}{\begin{tabular}[c]{@{}l@{}}Прямой ход; переход к следующему\\ предложению.\end{tabular}} \\ \hline
			\multicolumn{3}{|l|}{...} \\ \hline
			\multicolumn{1}{|r|}{18} & \multicolumn{1}{l|}{\begin{tabular}[c]{@{}l@{}}Сравнение FindPersonByBrandColor("Ford", "White", Name, City, PhoneNumber, BankName). \\ и\\ FindPersonByBrandColor(CarBrand, CarColor, Name, City, PhoneNumber, BankName)\\ Унификация успешна.\\ Подстановка:\{ CarBrand = "Ford",\\ CarColor = "White"\}\end{tabular}} & \multicolumn{1}{l|}{\begin{tabular}[c]{@{}l@{}}Прямой ход; переход к терму правила.\\ Унификация\\ UsingCar(Name, CarBrand, CarColor, \_)\end{tabular}} \\ \hline
			\multicolumn{1}{|r|}{19} & \multicolumn{1}{l|}{\begin{tabular}[c]{@{}l@{}}Сравнение UsingCar(Name, "Ford", "White", \_) и\\ UsingNumber("Kazaeva", "88126152221", address("Moscow", "Baumanskaya st.", 69, 1337)).\\ Унификация неуспешна.\end{tabular}} & \multicolumn{1}{l|}{\begin{tabular}[c]{@{}l@{}}Прямой ход; переход к следующему\\ предложению.\end{tabular}} \\ \hline
			\multicolumn{3}{|l|}{...} \\ \hline
			\multicolumn{1}{|r|}{26} & \multicolumn{1}{l|}{\begin{tabular}[c]{@{}l@{}}Сравнение UsingCar(Name, "Ford", "White", \_) и\\ UsingCar("Paraskun", "Ford", "White", 46190).\\ Унификация успешна.\\ Подстановка: \{ Surname = "Paraskun"\}\end{tabular}} & \multicolumn{1}{l|}{\begin{tabular}[c]{@{}l@{}}Обратный ход; переход к следующему\\ терму правила\\ UsingNumber(Name, PhoneNumber, address(City, \_, \_, \_))\end{tabular}} \\ \hline
			\multicolumn{1}{|r|}{27} & \multicolumn{1}{l|}{\begin{tabular}[c]{@{}l@{}}Сравнение UsingNumber("Paraskun", PhoneNumber,\\ address(City, \_, \_, \_)) и \\ UsingNumber("Kazaeva", "88126152221", address("Moscow", "Baumanskaya st.", 69, 1337)).\\ Унификация неуспешна.\end{tabular}} & \multicolumn{1}{l|}{\begin{tabular}[c]{@{}l@{}}Прямой ход; переход к следующему\\ предложению.\end{tabular}} \\ \hline
			\multicolumn{3}{|l|}{...} \\ \hline
			\multicolumn{1}{|r|}{29} & \multicolumn{1}{l|}{\begin{tabular}[c]{@{}l@{}}Сравнение UsingNumber("Paraskun", PhoneNumber,\\ address(City, \_, \_, \_)) и \\ UsingNumber("Paraskun", "83452878650", address("Moscow", "Ladojskaya st.", 1488, 666)).\\ Унификация успешна.\\ Подстановка: \{Phone\_number="83452878650",\\ City="Moscow"\}\end{tabular}} & \multicolumn{1}{l|}{\begin{tabular}[c]{@{}l@{}}Обратный ход; переход к следующему\\ терму правила\\ BankDepositor(Name, BankName, \_, \_).\end{tabular}} \\ \hline
			\multicolumn{1}{|r|}{30} & \multicolumn{1}{l|}{\begin{tabular}[c]{@{}l@{}}Сравнение bankDepositor(“Paraskun”, BankName, \_, \_)\\ и \\ UsingNumber("Suprunova", "88638423840", address("Moscow", "Novaya Doroga st.", 999, 666)).\\ Унификация неуспешна.\end{tabular}} & \multicolumn{1}{l|}{\begin{tabular}[c]{@{}l@{}}Прямой ход; переход к следующему\\ предложению.\end{tabular}} \\ \hline
			\multicolumn{3}{|l|}{...} \\ \hline
			\multicolumn{1}{|r|}{36} & \multicolumn{1}{l|}{\begin{tabular}[c]{@{}l@{}}Сравнение bankDepositor(“Paraskun”, BankName, \_, \_)\\ и\\ BankDepositor("Paraskun", "Not Sberbank", "07279163", 6900000). \\ Унификация успешна.\\ Подстановка: \{ Bank\_name ="Not Sberbank"\}\end{tabular}} & \multicolumn{1}{l|}{\begin{tabular}[c]{@{}l@{}}Обратный ход; переход к следующему\\ терму правила; термов больше нет.\\ Обратный ход (база знаний закончилась).\\ Вывод результатов на экран.\end{tabular}} \\ \hline
		\end{tabular}
	}
\end{sidewaystable}

\begin{sidewaystable}
	\medskip
	\resizebox{\linewidth}{!}{%
		\tabcolsep=2pt
		\begin{tabular}{|ccc|}
			\hline
			\multicolumn{1}{|l|}{\# шага} & \multicolumn{1}{l|}{\begin{tabular}[c]{@{}l@{}}Сравниваемые термы; результат;\\ Подстановка, если есть\end{tabular}} & \begin{tabular}[c]{@{}l@{}}Дальнейшие действия: прямой ход или\\ откат (к чему приводит?)\end{tabular} \\ \hline
			\multicolumn{1}{|r|}{1} & \multicolumn{1}{l|}{\begin{tabular}[c]{@{}l@{}}Сравнение FindPersonByBrandColor("Ferrari", "Red", Name, City, PhoneNumber, BankName).\\ и UsingNumber("Kazaeva", "88126152221", address("Moscow", "Baumanskaya st.", 69, 1337)).\\ Унификация неуспешна.\end{tabular}} & \begin{tabular}[c]{@{}l@{}}Прямой ход; переход к следующему\\ предложению.\end{tabular} \\ \hline
			\multicolumn{3}{|l|}{...} \\ \hline
			\multicolumn{1}{|r|}{18} & \multicolumn{1}{l|}{\begin{tabular}[c]{@{}l@{}}Сравнение \\ FindPersonByBrandColor("Ferrari", "Red", Name, City, PhoneNumber, BankName).\\ и FindPersonByBrandColor(CarBrand, CarColor, Name, City, PhoneNumber, BankName)\\ Унификация успешна.\\ Подстановка:\{ Car\_brand = "Ferrari",\\ Car\_color = "Red"\}\end{tabular}} & \begin{tabular}[c]{@{}l@{}}Прямой ход; переход к терму правила.\\ Унификация\\ UsingCar(Name, CarBrand, CarColor, \_)\end{tabular} \\ \hline
			\multicolumn{3}{|l|}{...} \\ \hline
			\multicolumn{1}{|r|}{24} & \multicolumn{1}{l|}{\begin{tabular}[c]{@{}l@{}}Сравнение UsingCar(Name, "Ferrari", "Red", \_) и \\ UsingCar("Suprunova", "Ferrari", "Red", 625000).\\ Унификация успешна.\\ Подстановка: \{ Surname = "Suprunova"\}\end{tabular}} & \begin{tabular}[c]{@{}l@{}}Прямой ход. Добавление терма с\\ подстановкой в память. Переход к\\ следующему предложению.\end{tabular} \\ \hline
			\multicolumn{1}{|r|}{25} & \multicolumn{1}{l|}{\begin{tabular}[c]{@{}l@{}}Сравнение UsingCar(Name, "Ferrari", "Red", \_) и \\ UsingCar("Alferova", "Ferrari", "Red", 625000).\\ Унификация успешна.\\ Подстановка: \{ Surname = "Alferova"\}\end{tabular}} & \begin{tabular}[c]{@{}l@{}}Прямой ход. Добавление терма с\\ подстановкой в память. Переход к\\ следующему предложению.\end{tabular} \\ \hline
			\multicolumn{3}{|l|}{...} \\ \hline
			\multicolumn{1}{|r|}{34} & \multicolumn{1}{l|}{\begin{tabular}[c]{@{}l@{}}Сравнение UsingCar(Name, "Ferrari", "Red", \_) и \\ FindPersonByBrandColor(CarBrand, CarColor, Name, City, PhoneNumber, BankName) \\ Унификация неуспешна.\end{tabular}} & \begin{tabular}[c]{@{}l@{}}Обратный ход (база знаний закончилась);\\ переход к следующему терму правила\\ UsingNumber(Name, PhoneNumber, address(City, \_, \_, \_))\end{tabular} \\ \hline
			\multicolumn{1}{|r|}{35} & \multicolumn{1}{l|}{\begin{tabular}[c]{@{}l@{}}Сравнение UsingNumber("Alferova", PhoneNumber, address(City, \_, \_, \_)) \\ и UsingNumber("Kazaeva", "88126152221", address("Moscow", "Baumanskaya st.", 69, 1337)).\\ Унификация неуспешна.\end{tabular}} & \begin{tabular}[c]{@{}l@{}}Прямой ход; переход к следующему\\ предложению.\end{tabular} \\ \hline
			\multicolumn{3}{|l|}{...} \\ \hline
			\multicolumn{1}{|r|}{39} & \multicolumn{1}{l|}{\begin{tabular}[c]{@{}l@{}}Сравнение UsingNumber("Alferova", PhoneNumber, address(City, \_, \_, \_)) \\ и UsingNumber("Alferova", "84232958684", address("Moscow", "Bakuninskaya st.", 969, 696)).\\ Унификация успешна.\\ Подстановка \{PhoneNumber=84232958684, City="Moscow"\}.\end{tabular}} & \begin{tabular}[c]{@{}l@{}}Прямой ход; переход к следующему\\ предложению.\end{tabular} \\ \hline
			\multicolumn{3}{|l|}{...} \\ \hline
			\multicolumn{1}{|r|}{51} & \multicolumn{1}{l|}{\begin{tabular}[c]{@{}l@{}}Сравнение UsingNumber("Alferova", PhoneNumber, address(City, \_, \_, \_)) \\ и FindPersonByBrandColor(CarBrand, CarColor, Name, City, PhoneNumber, BankName) \\ Унификация неуспешна.\end{tabular}} & \begin{tabular}[c]{@{}l@{}}Обратный ход(база знаний закончилась);\\ переход к следующему терму правила\\ BankDepositor(Name, BankName, \_, \_).\end{tabular} \\ \hline
			\multicolumn{1}{|r|}{52} & \multicolumn{1}{l|}{\begin{tabular}[c]{@{}l@{}}Сравнение BankDepositor("Alferova", BankName, \_, \_).\\ и UsingNumber("Kazaeva", "88126152221", address("Moscow", "Baumanskaya st.", 69, 1337)).\\ Унификация неуспешна.\end{tabular}} & \begin{tabular}[c]{@{}l@{}}Прямой ход; переход к следующему\\ предложению.\end{tabular} \\ \hline
			\multicolumn{3}{|l|}{...} \\ \hline
		\end{tabular}
	}
\end{sidewaystable}

\begin{sidewaystable}
	\medskip
	\resizebox{\linewidth}{!}{%
		\tabcolsep=2pt
		\begin{tabular}{|ccc|}
			\hline
			\multicolumn{1}{|l|}{\# шага} & \multicolumn{1}{l|}{\begin{tabular}[c]{@{}l@{}}Сравниваемые термы; результат;\\ Подстановка, если есть\end{tabular}} & \begin{tabular}[c]{@{}l@{}}Дальнейшие действия: прямой ход или\\ откат (к чему приводит?)\end{tabular} \\ \hline
			\multicolumn{1}{|r|}{64} &  \multicolumn{1}{l|}{\begin{tabular}[c]{@{}l@{}}Сравнение BankDepositor("Alferova", BankName, \_, \_).\\ и BankDepositor("Alferova", "VTB", "41572869", 300000).\\ Унификация успешна.\\ Подстановка \{BankName = "VTB"\}\end{tabular}} & \begin{tabular}[c]{@{}l@{}}Прямой ход; переход к следующему\\ предложению\\ Сохранение в памяти для данного терма подстановки\end{tabular} \\ \hline
			\multicolumn{3}{|l|}{...} \\ \hline
			\multicolumn{1}{|r|}{69} & \multicolumn{1}{l|}{\begin{tabular}[c]{@{}l@{}}Сравнение BankDepositor("Alferova", BankName, \_, \_).\\ и FindPersonByBrandColor(CarBrand, CarColor, Name, City, PhoneNumber, BankName) \\ Унификация неуспешна.\end{tabular}} & \begin{tabular}[c]{@{}l@{}}Обратный ход(база знаний закончилась);\\ Возврат к предыдущему терму\\ UsingNumber("Suprunova", PhoneNumber, address(City, \_, \_, \_)) с подстановкой из шага 39;\\ обратный ход из шага 39;\end{tabular} \\ \hline
			\multicolumn{3}{|l|}{Аналогичные шаги 37-69, но для новой подстановки} \\ \hline
			\multicolumn{3}{|l|}{...} \\ \hline
			\multicolumn{1}{|r|}{104} & \multicolumn{1}{l|}{Аналогичные шаги 24-63 для новой подстановки} & \begin{tabular}[c]{@{}l@{}}Обратный ход (база знаний закончилась);\\ вывод результата на экран.\end{tabular} \\ \hline
		\end{tabular}
	}
\end{sidewaystable}


\begin{sidewaystable}
	\medskip
	\resizebox{\linewidth}{!}{%
		\tabcolsep=2pt
		\begin{tabular}{|ccc|}
			\hline
			\multicolumn{1}{|l|}{\# шага} & \multicolumn{1}{l|}{\begin{tabular}[c]{@{}l@{}}Сравниваемые термы; результат;\\ Подстановка, если есть\end{tabular}} & \begin{tabular}[c]{@{}l@{}}Дальнейшие действия: прямой ход или\\ откат (к чему приводит?)\end{tabular} \\ \hline
			\multicolumn{1}{|r|}{1} & \multicolumn{1}{l|}{\begin{tabular}[c]{@{}l@{}}Сравнение FindPersonByBrandColor("Ferrari", "Red", Name, City, PhoneNumber, BankName).\\ и UsingNumber("Kazaeva", "88126152221", address("Moscow", "Baumanskaya st.", 69, 1337)).\\ Унификация неуспешна.\end{tabular}} & \begin{tabular}[c]{@{}l@{}}Прямой ход; переход к следующему\\ предложению.\end{tabular} \\ \hline
			\multicolumn{3}{|l|}{...} \\ \hline
			\multicolumn{1}{|r|}{18} & \multicolumn{1}{l|}{\begin{tabular}[c]{@{}l@{}}Сравнение \\ FindPersonByBrandColor("Ferrari", "Red", Name, City, PhoneNumber, BankName).\\ и FindPersonByBrandColor(CarBrand, CarColor, Name, City, PhoneNumber, BankName)\\ Унификация успешна.\\ Подстановка:\{ Car\_brand = "Ferrari",\\ Car\_color = "Red"\}\end{tabular}} & \begin{tabular}[c]{@{}l@{}}Прямой ход; переход к терму правила.\\ Унификация\\ UsingCar(Name, CarBrand, CarColor, \_)\end{tabular} \\ \hline
			\multicolumn{3}{|l|}{...} \\ \hline
			\multicolumn{1}{|r|}{24} & \multicolumn{1}{l|}{\begin{tabular}[c]{@{}l@{}}Сравнение UsingCar(Name, "Ferrari", "Red", \_) и \\ UsingCar("Suprunova", "Ferrari", "Red", 625000).\\ Унификация успешна.\\ Подстановка: \{ Surname = "Suprunova"\}\end{tabular}} & \begin{tabular}[c]{@{}l@{}}Прямой ход. Добавление терма с\\ подстановкой в память. Переход к\\ следующему предложению.\end{tabular} \\ \hline
			\multicolumn{1}{|r|}{25} & \multicolumn{1}{l|}{\begin{tabular}[c]{@{}l@{}}Сравнение UsingCar(Name, "Ferrari", "Red", \_) и \\ UsingCar("Alferova", "Ferrari", "Red", 625000).\\ Унификация успешна.\\ Подстановка: \{ Surname = "Alferova"\}\end{tabular}} & \begin{tabular}[c]{@{}l@{}}Прямой ход. Добавление терма с\\ подстановкой в память. Переход к\\ следующему предложению.\end{tabular} \\ \hline
			\multicolumn{3}{|l|}{...} \\ \hline
			\multicolumn{1}{|r|}{34} & \multicolumn{1}{l|}{\begin{tabular}[c]{@{}l@{}}Сравнение UsingCar(Name, "Ferrari", "Red", \_) и \\ FindPersonByBrandColor(CarBrand, CarColor, Name, City, PhoneNumber, BankName) \\ Унификация неуспешна.\end{tabular}} & \begin{tabular}[c]{@{}l@{}}Обратный ход (база знаний закончилась);\\ переход к следующему терму правила\\ UsingNumber(Name, PhoneNumber, address(City, \_, \_, \_))\end{tabular} \\ \hline
			\multicolumn{1}{|r|}{35} & \multicolumn{1}{l|}{\begin{tabular}[c]{@{}l@{}}Сравнение UsingNumber("Alferova", PhoneNumber, address(City, \_, \_, \_)) \\ и UsingNumber("Kazaeva", "88126152221", address("Moscow", "Baumanskaya st.", 69, 1337)).\\ Унификация неуспешна.\end{tabular}} & \begin{tabular}[c]{@{}l@{}}Прямой ход; переход к следующему\\ предложению.\end{tabular} \\ \hline
			\multicolumn{3}{|l|}{...} \\ \hline
			\multicolumn{1}{|r|}{39} & \multicolumn{1}{l|}{\begin{tabular}[c]{@{}l@{}}Сравнение UsingNumber("Alferova", PhoneNumber, address(City, \_, \_, \_)) \\ и UsingNumber("Alferova", "84232958684", address("Moscow", "Bakuninskaya st.", 969, 696)).\\ Унификация успешна.\\ Подстановка \{PhoneNumber=84232958684, City="Moscow"\}.\end{tabular}} & \begin{tabular}[c]{@{}l@{}}Прямой ход; переход к следующему\\ предложению.\end{tabular} \\ \hline
			\multicolumn{3}{|l|}{...} \\ \hline
			\multicolumn{1}{|r|}{51} & \multicolumn{1}{l|}{\begin{tabular}[c]{@{}l@{}}Сравнение UsingNumber("Alferova", PhoneNumber, address(City, \_, \_, \_)) \\ и FindPersonByBrandColor(CarBrand, CarColor, Name, City, PhoneNumber, BankName) \\ Унификация неуспешна.\end{tabular}} & \begin{tabular}[c]{@{}l@{}}Обратный ход(база знаний закончилась);\\ переход к следующему терму правила\\ BankDepositor(Name, BankName, \_, \_).\end{tabular} \\ \hline
			\multicolumn{1}{|r|}{52} & \multicolumn{1}{l|}{\begin{tabular}[c]{@{}l@{}}Сравнение BankDepositor("Alferova", BankName, \_, \_).\\ и UsingNumber("Kazaeva", "88126152221", address("Moscow", "Baumanskaya st.", 69, 1337)).\\ Унификация неуспешна.\end{tabular}} & \begin{tabular}[c]{@{}l@{}}Прямой ход; переход к следующему\\ предложению.\end{tabular} \\ \hline	
		\end{tabular}
	}
\end{sidewaystable}

\begin{sidewaystable}
	\medskip
	\resizebox{\linewidth}{!}{%
		\tabcolsep=2pt
		\begin{tabular}{|ccc|}
			\hline
			\multicolumn{1}{|l|}{\# шага} & \multicolumn{1}{l|}{\begin{tabular}[c]{@{}l@{}}Сравниваемые термы; результат;\\ Подстановка, если есть\end{tabular}} & \begin{tabular}[c]{@{}l@{}}Дальнейшие действия: прямой ход или\\ откат (к чему приводит?)\end{tabular} \\ \hline
			\multicolumn{3}{|l|}{...} \\ \hline
			\multicolumn{1}{|r|}{64} & \multicolumn{1}{l|}{\begin{tabular}[c]{@{}l@{}}Сравнение BankDepositor("Alferova", BankName, \_, \_).\\ и BankDepositor("Alferova", "VTB", "41572869", 300000).\\ Унификация успешна.\\ Подстановка \{BankName = "VTB"\}\end{tabular}} & \begin{tabular}[c]{@{}l@{}}Прямой ход; переход к следующему\\ предложению\\ Сохранение в памяти для данного терма подстановки\end{tabular} \\ \hline
			\multicolumn{3}{|l|}{...} \\ \hline
			\multicolumn{1}{|r|}{69} & \multicolumn{1}{l|}{\begin{tabular}[c]{@{}l@{}}Сравнение BankDepositor("Alferova", BankName, \_, \_).\\ и FindPersonByBrandColor(CarBrand, CarColor, Name, City, PhoneNumber, BankName) \\ Унификация неуспешна.\end{tabular}} & \begin{tabular}[c]{@{}l@{}}Обратный ход(база знаний закончилась);\\ Возврат к предыдущему терму\\ UsingNumber("Suprunova", PhoneNumber, address(City, \_, \_, \_)) с подстановкой из шага 39;\\ обратный ход из шага 39;\end{tabular} \\ \hline
			\multicolumn{3}{|l|}{Аналогичные шаги 37-69, но для новой подстановки} \\ \hline
			\multicolumn{3}{|l|}{...} \\ \hline
			\multicolumn{1}{|r|}{104} & \multicolumn{1}{l|}{Аналогичные шаги 24-63 для новой подстановки} & \begin{tabular}[c]{@{}l@{}}Обратный ход (база знаний закончилась);\\ вывод результата на экран.\end{tabular} \\ \hline
		\end{tabular}
	}
\end{sidewaystable}
