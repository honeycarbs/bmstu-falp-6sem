\textsc{\huge Лабораторная работа 13} \\
Создать базу знаний «Собственники» , дополнив (и минимально изменив) базу
знаний, хранящую знания (лаб. 12) знаниями о дополнительной собственности владельца. Преобразовать знания об автомобиле к форме знаний о собственности.
Вид собственности (кроме автомобиля):
\begin{tasks}[label=\arabic*]
	\task Строение, стоимость и другие его характеристики;
	\task Участок, стоимость и другие его характеристики;
	\task Водный транспорт, стоимость и другие его характеристики.
\end{tasks}
Описать и использовать вариантный домен: Собственность. Владелец может иметь,
но только один объект каждого вида собственности (это касается и автомобиля), или не иметь некоторых видов собственности.

Используя конъюнктивное правило и
разные формы задания одного вопроса (пояснять для какого №задания – какой вопрос),
обеспечить возможность поиска:
\begin{tasks}[label=\arabic*]
	\task Названий всех объектов собственности заданного субъекта,
	\task Названий и стоимости всех объектов собственности заданного субъекта,
	\task * Разработать правило, позволяющее найти суммарную стоимость всех
	объектов собственности заданного субъекта.
\end{tasks}
Для 2-го пункт и одной фамилии составить таблицу, отражающую конкретный
порядок работы системы, с объяснениями порядка работы и особенностей использования
доменов.

\begin{lstlisting}
domains
  price = integer.

  name = symbol.
  phoneNumber = symbol.
  city = symbol.
  street = symbol.
  houseNumber = integer.
  flatNumber = integer.
  fullAddress = address(city, street, houseNumber, flatNumber)

  carBrand = symbol.
  carColor = symbol.
  carCost = integer.

  bankName = symbol.
  bankAccount = symbol.
  accountCost = integer.

  propertyName = symbol.
  type = symbol.
  property = building(propertyName, type, price);
    land(propertyName, type, price);
    waterTransport(propertyName, type, price);
    car(propertyName, carBrand, carColor, price).
predicates
  UsingNumber(name, phoneNumber, fullAddress).
  BankDepositor(name, bankName, bankAccount, accountCost).
  Ownes(name, property).

  FindPersonByNameProperty(name, name, propertyName).
  FindPersonByNamePriceProperty(name, name, propertyName, price).

  FindSum(name, price).
  FindSumPriceProperties(name, price).
  IfHasProperty(name, type, price).

clauses
  UsingNumber("Kazaeva", "88126152221", address("Moscow", "Baumanskaya st.", 69, 1337)).
  UsingNumber("Kazaeva", "88123616141", address("St. Petersburg", "Not Baumansksya st.", 68, 1336)).
  UsingNumber("Paraskun", "83452878650", address("Moscow", "Ladojskaya st.", 1488, 666)).
  UsingNumber("Suprunova", "88638423840", address("Moscow", "Novaya Doroga st.", 999, 666)).
  UsingNumber("Alferova", "84232958684", address("Moscow", "Bakuninskaya st.", 969, 696)).

  Ownes("Kazaeva", car("MyBMW", "BMW", "Pink", 35700)).
  Ownes("Suprunova", car("Vot eto tachka", "Ferrari", "Red", 625000)).
  Ownes("Paraskun", building("Vot eto villa", "Villa near Pacific Ocean", 1488000000)).
  Ownes("Kazaeva", land("Island in SIms", "Sims Lot", 500)).
  Ownes("Alferova", waterTransport("MyPinkPig", "Canoe", 69000000)).

  BankDepositor("Paraskun", "Not Sberbank", "07279163", 6900000).
  BankDepositor("Kazaeva", "Tinkoff", "50679823", 3).
  BankDepositor("Alferova", "VTB", "41572869", 300000).
  BankDepositor("Suprunova", "Dorama Bank", "92033800", 50000).

  IfHasProperty(Name, car, Price) :-  Ownes(Name, car(_, _, _, Price)), !.
  IfHasProperty(Name, building, Price) :-  Ownes(Name, building(_, _, Price)), !.
  IfHasProperty(Name, waterTransport, Price) :-  Ownes(Name, waterTransport(_, _, Price)), !.
  IfHasProperty(Name, land, Price) :-  Ownes(Name, land(_, _, Price)), !.
  IfHasProperty(_,_,0).

  FindSum(Name, SumPrice) :- IfHasProperty(Name, car, CarPrice), IfHasProperty(Name, building, BuildPrice), IfHasProperty(Name, waterTransport, WaterPrice), IfHasProperty(Name, land, LandPrice),
  SumPrice = CarPrice + BuildPrice + WaterPrice + LandPrice.

  FindPersonByNameProperty(Name, Name, PropertyName) :- Ownes(Name, car(PropertyName, _, _, _)).
  FindPersonByNameProperty(Name, Name, PropertyName) :- Ownes(Name, building(PropertyName, _, _)).
  FindPersonByNameProperty(Name, Name, PropertyName) :- Ownes(Name, waterTransport(PropertyName, _, _)).
  FindPersonByNameProperty(Name, Name, PropertyName) :- Ownes(Name, land(PropertyName, _, _)).

  FindPersonByNamePriceProperty(Name, Name, PropertyName, Price) :- Ownes(Name, car(PropertyName, _, _, Price)).
  FindPersonByNamePriceProperty(Name, Name, PropertyName, Price) :- Ownes(Name, building(PropertyName, _, Price)).
  FindPersonByNamePriceProperty(Name, Name, PropertyName, Price) :- Ownes(Name, waterTransport(PropertyName, _, Price)).
  FindPersonByNamePriceProperty(Name, Name, PropertyName, Price) :- Ownes(Name, land(PropertyName, _, Price)).

  FindSumPriceProperties(Name, SumPrice) :- findSum(Name, SumPrice).	
goal
  FindPersonByNamePriceProperty("Kazaeva", Name, PropertyName, Price).
  %FindSumPriceProperties("Kazaeva", SumPrice).
\end{lstlisting}

\begin{sidewaystable}
	\medskip
	\resizebox{\linewidth}{!}{%
		\tabcolsep=2pt
		\begin{tabular}{|ccc|}
			\hline
			\multicolumn{1}{|c|}{\# шага} & \multicolumn{1}{c|}{\begin{tabular}[c]{@{}c@{}}Сравниваемые термы; результат;\\ Подстановка, если есть\end{tabular}} & \begin{tabular}[c]{@{}c@{}}Дальнейшие действия: прямой ход или\\ откат (к чему приводит?)\end{tabular} \\ \hline
			\multicolumn{1}{|c|}{1} & \multicolumn{1}{c|}{\begin{tabular}[c]{@{}c@{}}Попытка унификации:\\ FindPersonByNamePriceProperty("Kazaeva", Name, PropertyName, Price).\\ UsingNumber("Kazaeva", "88126152221", address("Moscow", "Baumanskaya st.", 69, 1337)).\\ Унификация неуспешна\end{tabular}} & \begin{tabular}[c]{@{}c@{}}Прямой ход; переход к следующему\\ предложению.\end{tabular} \\ \hline
			\multicolumn{3}{|c|}{...} \\ \hline
			\multicolumn{1}{|c|}{24} & \multicolumn{1}{c|}{\begin{tabular}[c]{@{}c@{}}Попытка унификации:\\ FindPersonByNamePriceProperty("Kazaeva", Name, PropertyName, Price).\\ FindPersonByNamePriceProperty(Name, Name, PropertyName, Price)\\ Унификация успешна.\\ Подстановка \{Name = "Kazaeva"\}\end{tabular}} & \begin{tabular}[c]{@{}c@{}}Прямой ход; переход к терму правила:\\ Ownes(Name, car(PropertyName, \_, \_, Price)).\\ Конкретизация Surname значением\\ "Kazaeva".\end{tabular} \\ \hline
			\multicolumn{3}{|c|}{...} \\ \hline
			\multicolumn{1}{|c|}{30} & \multicolumn{1}{c|}{\begin{tabular}[c]{@{}c@{}}Попытка унификации:\\ Ownes("Kazaeva", car(PropertyName, \_, \_, Price)).\\ Ownes("Kazaeva", car("MyBMW", "BMW", "Pink", 35700)).\\ Унификация успешна.\\ Подстановка \{ProperyName = "MyBMW", Price="35700"\}.\end{tabular}} & \begin{tabular}[c]{@{}c@{}}Прямой ход; переход к следующему\\ предложению.\\ Сохранение результата в памяти.\end{tabular} \\ \hline
			\multicolumn{3}{|c|}{...} \\ \hline
			\multicolumn{1}{|c|}{33} & \multicolumn{1}{c|}{\begin{tabular}[c]{@{}c@{}}Попытка унификации:\\ Ownes("Kazaeva", car(PropertyName, \_, \_, Price)).\\ Ownes("Kazaeva", land("Island in SIms", "Sims Lot", 500)).\\ Унификация неуспешна.\end{tabular}} & \begin{tabular}[c]{@{}c@{}}Прямой ход; переход к следующему\\ предложению.\end{tabular} \\ \hline
			\multicolumn{3}{|c|}{...} \\ \hline
			\multicolumn{1}{|c|}{54} & \multicolumn{1}{c|}{\begin{tabular}[c]{@{}c@{}}Попытка унификации:\\ Ownes("Kazaeva", car(PropertyName, \_, \_, Price)).\\ FindSumPriceProperties(Name, SumPrice) :- findSum(Name, SumPrice).\\ Унификация неуспешна.\end{tabular}} & \begin{tabular}[c]{@{}c@{}}Обратный ход, просмотрена вся БЗ;\\ переход к терму,\\ следующему после терма из шага 24:\\ fFindPersonByNamePriceProperty(Name, Name, PropertyName, Price) :- Ownes(Name, building(PropertyName, \_, Price)).\end{tabular} \\ \hline
			\multicolumn{1}{|c|}{55} & \multicolumn{1}{c|}{\begin{tabular}[c]{@{}c@{}}Попытка унификации:\\ FindPersonByNamePriceProperty("Kazaeva", Name, PropertyName, Price) и\\ FindPersonByNamePriceProperty(Name, Name, PropertyName, Price).\\ Унификация успешна.\\ Подстановка: \{Surname = "Kazaeva"\}.\end{tabular}} & \begin{tabular}[c]{@{}c@{}}Прямой ход; переход к терму правила:\\ Ownes(Name, building(PropertyName, \_, Price)).\\ Конкретизация Surname значением\\ "Kazaeva".\end{tabular} \\ \hline
			\multicolumn{3}{|c|}{...} \\ \hline
			\multicolumn{1}{|c|}{84} & \multicolumn{1}{c|}{\begin{tabular}[c]{@{}c@{}}Попытка унификации:\\ Ownes("Kazaeva", building(PropertyName, \_, Price)). и\\ FindSumPriceProperties(Name, SumPrice) :- findSum(Name, SumPrice).\\ Унификация неуспешна.\end{tabular}} & \begin{tabular}[c]{@{}c@{}}Обратный ход, просмотрена вся БЗ -\\ переход к следующему терму,\\ следующему после терма из шага 54:\\ FindPersonByNamePriceProperty(Name, Name, PropertyName, Price) :- Ownes(Name, waterTransport(PropertyName, \_, Price)).\end{tabular} \\ \hline
			\multicolumn{3}{|c|}{...} \\ \hline
			\multicolumn{1}{|c|}{113} & \multicolumn{1}{c|}{\begin{tabular}[c]{@{}c@{}}Попытка унификации:\\ Ownes("Kazaeva", waterTransport(PropertyName, \_, Price)). и\\ FindSumPriceProperties(Name, SumPrice) :- findSum(Name, SumPrice).\\ Унификация неуспешна.\end{tabular}} & \begin{tabular}[c]{@{}c@{}}Обратный ход, просмотрена вся БЗ -\\ переход к следующему терму,\\ следующему после терма из шага 84:\\ FindPersonByNamePriceProperty(Name, Name, PropertyName, Price) :- Ownes(Name, land(PropertyName, \_, Price)).\end{tabular} \\ \hline
			\multicolumn{3}{|c|}{...} \\ \hline
			\multicolumn{1}{|c|}{122} & \multicolumn{1}{c|}{\begin{tabular}[c]{@{}c@{}}Попытка унификации:\\ Ownes("Kazaeva", land(PropertyName, \_, Price)). и\\ Ownes("Kazaeva", land("Island in SIms", "Sims Lot", 500)).\\ Унификация успешна.\\ \{PropertyName = "Island in SIms", Price = 500\}.\end{tabular}} & \begin{tabular}[c]{@{}c@{}}Прямой ход; переход к следующему\\ предложению.\\ Сохранение результата в памяти.\end{tabular} \\ \hline
			\multicolumn{3}{|c|}{...} \\ \hline
			\multicolumn{1}{|c|}{145} & \multicolumn{1}{c|}{\begin{tabular}[c]{@{}c@{}}Попытка унификации:\\ Ownes("Kazaeva", land(PropertyName, \_, Price)).  и\\ FindSumPriceProperties(Name, SumPrice) :- findSum(Name, SumPrice).\\ Унификация неуспешна.\end{tabular}} & \begin{tabular}[c]{@{}c@{}}Обратный ход (просмотрена вся БЗ);\\ Вывод результатов.\end{tabular} \\ \hline
		\end{tabular}
	}
\end{sidewaystable}