\chapter{Практические задания}
\textit{Используя функционалы:}
\begin{enumerate}[wide=0pt]
\item \textit{Напишите функцию, которая уменьшает на 10 все числа из списка-аргумента этой функции.}
\begin{lstlisting}
(defun subtract-dec(lst)
  (mapcar #'(lambda (elem) (cond ((numberp elem)(- elem 10))
                                  ((listp elem)(subtract-dec elem))
                                  (t elem))) lst))
\end{lstlisting}
\item \textit{Напишите функцию, которая умножает на заданное число-аргумент все числа из заданного списка-аргумента, когда:}
\begin{enumerate}
	\item \textit{все элементы списка --- числа,}
	\item \textit{элементы списка -- любые объекты.}
\end{enumerate}
\begin{lstlisting}
(defun mul-all(m lst)
  (mapcar #'(lambda (elem) (cond ((numberp elem)(* m elem))
                                  ((listp elem)(mul-all m elem))
                                  (t elem))) lst))
\end{lstlisting}
\item \textit{Написать функцию, которая по своему списку-аргументу lst определяет является ли он палиндромом (то есть равны ли lst и (reverse lst)).}
\begin{lstlisting}
(defun polyp(lst) 
  (equal lst (reverse lst)))
\end{lstlisting}
\item \textit{Написать предикат set-equal, который возвращает t, если два его \\ множества-аргумента содержат одни и те же элементы, порядок которых не имеет значения}
\begin{lstlisting}
(defun set-equalp (set1 set2)
(and (= (length set1) (length set2))
     (every #'(lambda (elem) (member elem set2 :test #'equal)) set1)
     (every #'(lambda (elem) (member elem set1 :test #'equal)) set2)))
\end{lstlisting}
\item \textit{Написать функцию которая получает как аргумент список чисел, а возвращает список квадратов этих чисел в том же порядке.}
\begin{lstlisting}
(defun make-square(lst)
  (mapcar #'(lambda (elem) (* elem elem)) lst))
\end{lstlisting}
\item \textit{Напишите функцию, select-between, которая из списка-аргумента, содержащего только числа, выбирает только те, которые расположены между двумя указанными границами-аргументами и возвращает их в виде списка (упорядоченного по возрастанию списка чисел (+ 2 балла)).}
\begin{lstlisting}
(defun select-between (lst left right)
  (sort (reduce #'(lambda (res el) (if (and (> el left) (< el right))
                                       (cons el res)
	                                   res)) lst :initial-value ()) #'<))
\end{lstlisting}
\item \textit{Написать функцию, вычисляющую декартово произведение двух своих списков-аргументов. (Напомним, что А х В это множество всевозможных пар (a b), где а принадлежит А, принадлежит В.)}
\begin{lstlisting}
(defun combinations(lst1 lst2)
  (mapcan #'(lambda (inner)
    (mapcar #'(lambda (outer)(list outer inner)) lst1)) lst2))
\end{lstlisting}
\item \textit{Почему так реализовано reduce, в чем причина?}
\begin{enumerate}
	\item \lstinline|(reduce #'+ 0) -> 0|
	\item \lstinline|(reduce #'+ 0) -> (reduce #'+ ()) -> 0|
\end{enumerate}
\item \textit{Пусть list-of-list список, состоящий из списков. Написать функцию, которая вычисляет сумму длин всех элементов list-of-list, т.е. например для аргумента ((1 2) (3 4)) -> 4}
\begin{lstlisting}
(defun inner-len(lists)
  (apply #'+ (mapcar #'(lambda (elem)(cond ((listp elem) (inner-len elem))
                                            (t 1))) lists)))
\end{lstlisting}
\end{enumerate}