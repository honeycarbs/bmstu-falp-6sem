\chapter{Практические задания}
\textit{Используя рекурсию:}
\begin{enumerate}[wide=0pt]
\item \textit{Написать хвостовую рекурсивную функцию my-reverse, которая развернет верхний уровень своего списка-аргумента lst.}
\begin{lstlisting}
(defun my-reverse(lst &optional (acc ()))
  (cond ((null lst) acc)
        ((consp lst) (my-reverse (cdr lst) (cons (car lst) acc)))))	
\end{lstlisting}

\item \textit{ Написать функцию, которая возвращает первый элемент \\ списка-аргумента, который сам является непустым списком.}
\begin{lstlisting}
(defun get-fst-list(lst)
  (cond ((null lst) nil)
        ((listp (car lst)) (car lst))
        (t (get-fst-list (cdr lst)))))
\end{lstlisting}

\item \textit{Написать функцию, которая выбирает из заданного списка только те числа, которые больше 1 и меньше 10.}
\begin{lstlisting}
(defun find-between(lst l r &optional (dump ()))
  (cond ((null lst) dump)
        (t (cond ((and (> (car lst) l) (< (car lst) r))
                  (find-between (cdr lst) l r (cons (car lst) dump)))
                 (t (find-between (cdr lst) l r dump))))))
\end{lstlisting}

\item \textit{Напишите рекурсивную функцию, которая умножает на заданное число-аргумент все числа из заданного списка-аргумента, когда}
\begin{enumerate}
	\item \textit{все элементы списка -- числа,}
	\item \textit{элементы списка -- любые объекты.} \newpage
\end{enumerate}
\begin{lstlisting}
;; ================================ (a) =================================
(defun multiply-numbers(lst m)
  (cond ((null lst) nil)
         (t (cons (* (car lst) m) (multiply-numbers (cdr lst) m)))))
;; ================================ (b) =================================
(defun multiply-objcts(lst m)
  (cond ((null lst) nil)
        ((numberp (car lst)) 
         (cons (* (car lst) m) (multiply-objcts (cdr lst) m)))
        ((listp (car lst)) 
         (cons (multiply-objcts (car lst) m) (multiply-objcts (cdr lst) m)))
        (t (cons (car lst) (multiply-objcts (cdr lst) m)))))
\end{lstlisting}
\item \textit{Напишите функцию, select-between, которая из списка-аргумента, содержащего только числа, выбирает только те, которые расположены между двумя указанными границами-аргументами и возвращает их в виде списка (упорядоченного по возрастанию списка чисел)}
\begin{lstlisting}
(defun select-between(lst l r &optional (dump ()))
  (sort (cond ((null lst) dump)
               (t (cond ((and (> (car lst) l) (< (car lst) r))
                         (select-between (cdr lst) l r (cons (car lst) dump)))
                   (t (select-between (cdr lst) l r dump))))) #'<))
\end{lstlisting}
\item \textit{Написать рекурсивную версию (с именем rec-add) вычисления суммы чисел заданного списка:}
\begin{enumerate}
	\item \textit{одноуровнего смешанного,}
	\item \textit{структурированного.}
\end{enumerate}
\begin{lstlisting}
;; ================================ (a) =================================
(defun rec-add-onelevel (lst &optional (acc 0))
  (cond ((null lst) acc)
        ((numberp (car lst)) (rec-add-onelevel (cdr lst) (+ acc (car lst))))
        (t (rec-add-onelevel (cdr lst) acc))))

;; ================================ (b) =================================
(defun rec-add-structured (lst &optional (acc 0))
  (cond ((null lst) acc)
        ((numberp (car lst)) (rec-add-structured (cdr lst) (+ acc (car lst))))
        ((listp (car lst))
                (rec-add-structured (cdr lst)
                                    (rec-add-structured (car lst) acc)))
        (t (rec-add-structured (cdr lst) acc))))
\end{lstlisting}
\item \textit{Написать рекурсивную версию с именем recnth функции nth.}
\begin{lstlisting}
(defun recnth(lst n &optional (acc 0))
  (cond ((< n 0) nil)
        ((null lst) nil)
        ((= acc n) (car lst))
        (t (recnth (cdr lst) n (+ 1 acc)))))
\end{lstlisting}
\item \textit{Написать рекурсивную функцию allodd, которая возвращает t когда все элементы списка нечетные.}
\begin{lstlisting}
(defun allodd(lst)
  (cond ((null lst) t)
        ((oddp (car lst)) (allodd (cdr lst)))
        (t nil)))
\end{lstlisting}
\item \textit{ Написать рекурсивную функцию, которая возвращает первое нечетное число из списка (структурированного), возможно создавая некоторые вспомогательные функции.}
\begin{lstlisting}
(defun get-fst-odd(lst)
  (cond ((null lst) nil)
        ((listp (car lst)) (let ((fnd (get-fst-odd (car lst))))
                                (if (not fnd)
                                    (get-fst-odd (cdr lst))
                                    fnd)))
        ((oddp (car lst)) (car lst))
        (t (get-fst-odd (cdr lst)))))
\end{lstlisting}
\item \textit{Используя cons-дополняемую рекурсию с одним тестом завершения, написать функцию которая получает как аргумент список чисел, а возвращает список квадратов этих чисел в том же порядке.}
\begin{lstlisting}
(defun get-squares (lst)
  (cond ((null lst) nil)
        (t (cons (* (car lst) (car lst)) (get-squares (cdr lst))))))
\end{lstlisting}
\end{enumerate}